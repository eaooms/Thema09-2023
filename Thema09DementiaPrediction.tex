% Options for packages loaded elsewhere
\PassOptionsToPackage{unicode}{hyperref}
\PassOptionsToPackage{hyphens}{url}
%
\documentclass[
]{article}
\usepackage{amsmath,amssymb}
\usepackage{lmodern}
\usepackage{iftex}
\ifPDFTeX
  \usepackage[T1]{fontenc}
  \usepackage[utf8]{inputenc}
  \usepackage{textcomp} % provide euro and other symbols
\else % if luatex or xetex
  \usepackage{unicode-math}
  \defaultfontfeatures{Scale=MatchLowercase}
  \defaultfontfeatures[\rmfamily]{Ligatures=TeX,Scale=1}
\fi
% Use upquote if available, for straight quotes in verbatim environments
\IfFileExists{upquote.sty}{\usepackage{upquote}}{}
\IfFileExists{microtype.sty}{% use microtype if available
  \usepackage[]{microtype}
  \UseMicrotypeSet[protrusion]{basicmath} % disable protrusion for tt fonts
}{}
\makeatletter
\@ifundefined{KOMAClassName}{% if non-KOMA class
  \IfFileExists{parskip.sty}{%
    \usepackage{parskip}
  }{% else
    \setlength{\parindent}{0pt}
    \setlength{\parskip}{6pt plus 2pt minus 1pt}}
}{% if KOMA class
  \KOMAoptions{parskip=half}}
\makeatother
\usepackage{xcolor}
\usepackage[margin=1in]{geometry}
\usepackage{color}
\usepackage{fancyvrb}
\newcommand{\VerbBar}{|}
\newcommand{\VERB}{\Verb[commandchars=\\\{\}]}
\DefineVerbatimEnvironment{Highlighting}{Verbatim}{commandchars=\\\{\}}
% Add ',fontsize=\small' for more characters per line
\usepackage{framed}
\definecolor{shadecolor}{RGB}{248,248,248}
\newenvironment{Shaded}{\begin{snugshade}}{\end{snugshade}}
\newcommand{\AlertTok}[1]{\textcolor[rgb]{0.94,0.16,0.16}{#1}}
\newcommand{\AnnotationTok}[1]{\textcolor[rgb]{0.56,0.35,0.01}{\textbf{\textit{#1}}}}
\newcommand{\AttributeTok}[1]{\textcolor[rgb]{0.77,0.63,0.00}{#1}}
\newcommand{\BaseNTok}[1]{\textcolor[rgb]{0.00,0.00,0.81}{#1}}
\newcommand{\BuiltInTok}[1]{#1}
\newcommand{\CharTok}[1]{\textcolor[rgb]{0.31,0.60,0.02}{#1}}
\newcommand{\CommentTok}[1]{\textcolor[rgb]{0.56,0.35,0.01}{\textit{#1}}}
\newcommand{\CommentVarTok}[1]{\textcolor[rgb]{0.56,0.35,0.01}{\textbf{\textit{#1}}}}
\newcommand{\ConstantTok}[1]{\textcolor[rgb]{0.00,0.00,0.00}{#1}}
\newcommand{\ControlFlowTok}[1]{\textcolor[rgb]{0.13,0.29,0.53}{\textbf{#1}}}
\newcommand{\DataTypeTok}[1]{\textcolor[rgb]{0.13,0.29,0.53}{#1}}
\newcommand{\DecValTok}[1]{\textcolor[rgb]{0.00,0.00,0.81}{#1}}
\newcommand{\DocumentationTok}[1]{\textcolor[rgb]{0.56,0.35,0.01}{\textbf{\textit{#1}}}}
\newcommand{\ErrorTok}[1]{\textcolor[rgb]{0.64,0.00,0.00}{\textbf{#1}}}
\newcommand{\ExtensionTok}[1]{#1}
\newcommand{\FloatTok}[1]{\textcolor[rgb]{0.00,0.00,0.81}{#1}}
\newcommand{\FunctionTok}[1]{\textcolor[rgb]{0.00,0.00,0.00}{#1}}
\newcommand{\ImportTok}[1]{#1}
\newcommand{\InformationTok}[1]{\textcolor[rgb]{0.56,0.35,0.01}{\textbf{\textit{#1}}}}
\newcommand{\KeywordTok}[1]{\textcolor[rgb]{0.13,0.29,0.53}{\textbf{#1}}}
\newcommand{\NormalTok}[1]{#1}
\newcommand{\OperatorTok}[1]{\textcolor[rgb]{0.81,0.36,0.00}{\textbf{#1}}}
\newcommand{\OtherTok}[1]{\textcolor[rgb]{0.56,0.35,0.01}{#1}}
\newcommand{\PreprocessorTok}[1]{\textcolor[rgb]{0.56,0.35,0.01}{\textit{#1}}}
\newcommand{\RegionMarkerTok}[1]{#1}
\newcommand{\SpecialCharTok}[1]{\textcolor[rgb]{0.00,0.00,0.00}{#1}}
\newcommand{\SpecialStringTok}[1]{\textcolor[rgb]{0.31,0.60,0.02}{#1}}
\newcommand{\StringTok}[1]{\textcolor[rgb]{0.31,0.60,0.02}{#1}}
\newcommand{\VariableTok}[1]{\textcolor[rgb]{0.00,0.00,0.00}{#1}}
\newcommand{\VerbatimStringTok}[1]{\textcolor[rgb]{0.31,0.60,0.02}{#1}}
\newcommand{\WarningTok}[1]{\textcolor[rgb]{0.56,0.35,0.01}{\textbf{\textit{#1}}}}
\usepackage{graphicx}
\makeatletter
\def\maxwidth{\ifdim\Gin@nat@width>\linewidth\linewidth\else\Gin@nat@width\fi}
\def\maxheight{\ifdim\Gin@nat@height>\textheight\textheight\else\Gin@nat@height\fi}
\makeatother
% Scale images if necessary, so that they will not overflow the page
% margins by default, and it is still possible to overwrite the defaults
% using explicit options in \includegraphics[width, height, ...]{}
\setkeys{Gin}{width=\maxwidth,height=\maxheight,keepaspectratio}
% Set default figure placement to htbp
\makeatletter
\def\fps@figure{htbp}
\makeatother
\setlength{\emergencystretch}{3em} % prevent overfull lines
\providecommand{\tightlist}{%
  \setlength{\itemsep}{0pt}\setlength{\parskip}{0pt}}
\setcounter{secnumdepth}{-\maxdimen} % remove section numbering
\ifLuaTeX
  \usepackage{selnolig}  % disable illegal ligatures
\fi
\IfFileExists{bookmark.sty}{\usepackage{bookmark}}{\usepackage{hyperref}}
\IfFileExists{xurl.sty}{\usepackage{xurl}}{} % add URL line breaks if available
\urlstyle{same} % disable monospaced font for URLs
\hypersetup{
  pdftitle={Thema09DementiaPrediction},
  pdfauthor={Ewoud},
  hidelinks,
  pdfcreator={LaTeX via pandoc}}

\title{Thema09DementiaPrediction}
\author{Ewoud}
\date{2023-09-06}

\begin{document}
\maketitle

\hypertarget{introduction}{%
\subsection{Introduction}\label{introduction}}

The question this research is aiming to give an answer to is: \emph{How
accurate can a machine learning algorithm be, that predicts if a subject
has dementia using different clinical parameters?}

Dataset:
\url{https://www.kaggle.com/datasets/shashwatwork/dementia-prediction-dataset}

\hypertarget{codebook}{%
\subsection{Codebook}\label{codebook}}

\hypertarget{description-of-some-of-the-rows}{%
\subsubsection{Description of some of the
rows}\label{description-of-some-of-the-rows}}

MRI :

SES : Socioeconomic status as assessed by the Hollingshead Index of
Social Position and classified into categories from 1 (highest status)
to 5 (lowest status)

MMSE : Mini--Mental State Examination (MMSE) The Mini--Mental State
Examination (MMSE) or Folstein test is a 30-point questionnaire that is
used extensively in clinical and research settings to measure cognitive
impairment. It is commonly used in medicine and allied health to screen
for dementia. It is also used to estimate the severity and progression
of cognitive impairment and to follow the course of cognitive changes in
an individual over time; thus making it an effective way to document an
individual's response to treatment. The MMSE's purpose has been not, on
its own, to provide a diagnosis for any particular nosological entity.

Interpretations

Any score greater than or equal to 24 points (out of 30) indicates a
normal cognition. Below this, scores can indicate severe (≤9 points),
moderate (10--18 points) or mild (19--23 points) cognitive impairment.
The raw score may also need to be corrected for educational attainment
and age. That is, a maximal score of 30 points can never rule out
dementia. Low to very low scores correlate closely with the presence of
dementia, although other mental disorders can also lead to abnormal
findings on MMSE testing. The presence of purely physical problems can
also interfere with interpretation if not properly noted; for example, a
patient may be physically unable to hear or read instructions properly,
or may have a motor deficit that affects writing and drawing skills.

CDR : Clinical Dementia Rating (CDR) The CDR™ in one aspect is a 5-point
scale used to characterize six domains of cognitive and functional
performance applicable to Alzheimer disease and related dementias:
Memory, Orientation, Judgment \& Problem Solving, Community Affairs,
Home \& Hobbies, and Personal Care. The necessary information to make
each rating is obtained through a semi-structured interview of the
patient and a reliable informant or collateral source (e.g., family
member) referred to as the CDR™ Assessment Protocol.

The CDR™ Scoring Table provides descriptive anchors that guide the
clinician in making appropriate ratings based on interview data and
clinical judgment. In addition to ratings for each domain, an overall
CDR™ score may be calculated through the use of an CDR™ Scoring
Algorithm. This score is useful for characterizing and tracking a
patient's level of impairment/dementia:

0 = Normal 0.5 = Very Mild Dementia 1 = Mild Dementia 2 = Moderate
Dementia 3 = Severe Dementia

eTIV: Estimated total intracranial volume (eTIV) The ICV measure,
sometimes referred to as total intracranial volume (TIV), refers to the
estimated volume of the cranial cavity as outlined by the supratentorial
dura matter or cerebral contour when dura is not clearly detectable. ICV
is often used in studies involved with analysis of the cerebral
structure under different imaging modalities, such as Magnetic Resonance
(MR), MR and Diffusion Tensor Imaging (DTI), MR and Single-photon
Emission Computed Tomography (SPECT), Ultrasound and Computed Tomography
(CT). ICV consistency during aging makes it a reliable tool for
correction of head size variation across subjects in studies that rely
on morphological features of the brain. ICV, along with age and gender
are reported as covariates to adjust for regression analyses in
investigating progressive neurodegenerative brain disorders, such as
Alzheimer's disease, aging and cognitive impairment. ICV has also been
utilized as an independent voxel based morphometric feature to evaluate
age-related changes in the structure of premorbid brai, determine
characterizing atrophy patterns in subjects with mild cognitive
impairment (MCI) and Alzheimer's disease (AD), delineate structural
abnormalities in the white matter (WM) in schizophrenia, epilepsy, and
gauge cognitive efficacy.

nWBV : Normalized whole-brain volume, expressed as a percent of all
voxels in the atlas-masked image that are labeled as gray or white
matter by the automated tissue segmentation process

ASF: Atlas scaling factor (unitless). Computed scaling factor that
transforms native-space brain and skull to the atlas target (i.e., the
determinant of the transform matrix)

\hypertarget{cleaning}{%
\subsection{Cleaning}\label{cleaning}}

First thing to do is to clean the dataset of zero values or outliers
that can obstruct this research

\hypertarget{missing-values}{%
\subsubsection{Missing values}\label{missing-values}}

Lets filter the rows out of the dataset with 0 or na.

\begin{Shaded}
\begin{Highlighting}[]
\NormalTok{Data1\_filtered }\OtherTok{\textless{}{-}}\NormalTok{ Data1 }\SpecialCharTok{\%\textgreater{}\%} \FunctionTok{drop\_na}\NormalTok{()}
\end{Highlighting}
\end{Shaded}

19 object are filtered

\hypertarget{finding-outliers}{%
\subsubsection{finding outliers}\label{finding-outliers}}

\begin{verbatim}
##   Subject ID           MRI ID             Group              Visit          
##  Length:354         Length:354         Length:354         Length:354        
##  Class :character   Class :character   Class :character   Class :character  
##  Mode  :character   Mode  :character   Mode  :character   Mode  :character  
##                                                                             
##                                                                             
##                                                                             
##     MR Delay          M/F                Hand                Age       
##  Min.   :   0.0   Length:354         Length:354         Min.   :60.00  
##  1st Qu.:   0.0   Class :character   Class :character   1st Qu.:71.00  
##  Median : 559.5   Mode  :character   Mode  :character   Median :77.00  
##  Mean   : 601.4                                         Mean   :77.03  
##  3rd Qu.: 882.5                                         3rd Qu.:82.00  
##  Max.   :2639.0                                         Max.   :98.00  
##       EDUC            SES            MMSE            CDR              eTIV     
##  Min.   : 6.00   Min.   :1.00   Min.   : 4.00   Min.   :0.0000   Min.   :1106  
##  1st Qu.:12.00   1st Qu.:2.00   1st Qu.:27.00   1st Qu.:0.0000   1st Qu.:1358  
##  Median :15.00   Median :2.00   Median :29.00   Median :0.0000   Median :1470  
##  Mean   :14.70   Mean   :2.46   Mean   :27.41   Mean   :0.2712   Mean   :1490  
##  3rd Qu.:16.75   3rd Qu.:3.00   3rd Qu.:30.00   3rd Qu.:0.5000   3rd Qu.:1595  
##  Max.   :23.00   Max.   :5.00   Max.   :30.00   Max.   :2.0000   Max.   :2004  
##       nWBV             ASF        
##  Min.   :0.6444   Min.   :0.8755  
##  1st Qu.:0.6987   1st Qu.:1.1002  
##  Median :0.7291   Median :1.1935  
##  Mean   :0.7299   Mean   :1.1938  
##  3rd Qu.:0.7569   3rd Qu.:1.2923  
##  Max.   :0.8368   Max.   :1.5873
\end{verbatim}

No visible outliers

\hypertarget{remove-colums-with-no-meaning}{%
\subsubsection{Remove colums with no
meaning}\label{remove-colums-with-no-meaning}}

\includegraphics{Thema09DementiaPrediction_files/figure-latex/unnamed-chunk-6-1.pdf}
When looking at the Handedness parameter we can see that the only
variable is R for right handed people, because everybody is right handed
we can remove the column

removing the MRI id and delay because these are not parameters that have
influence on the outcome if a subject has dementia

CDR is basically a parameter telling if a patient has dementia or not so
it will not be taking in for machine learning but testing if algorithm
is correct.

\hypertarget{testing-the-dataset}{%
\subsection{Testing the dataset}\label{testing-the-dataset}}

Second thing to do is to look at the underlying distribution and the
variation within the dataset

\hypertarget{equal-distribution}{%
\subsubsection{equal distribution}\label{equal-distribution}}

\includegraphics{Thema09DementiaPrediction_files/figure-latex/unnamed-chunk-9-1.pdf}

\hypertarget{finding-correlation}{%
\subsection{Finding correlation}\label{finding-correlation}}

\includegraphics{Thema09DementiaPrediction_files/figure-latex/unnamed-chunk-10-1.pdf}
\includegraphics{Thema09DementiaPrediction_files/figure-latex/unnamed-chunk-10-2.pdf}
\includegraphics{Thema09DementiaPrediction_files/figure-latex/unnamed-chunk-10-3.pdf}
Waardes die waarschijnlijk van invloed zijn Educatie, SES, MMSE, nWBV,
klein beetje eTIV en ASFs

\end{document}
