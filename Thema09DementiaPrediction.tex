% Options for packages loaded elsewhere
\PassOptionsToPackage{unicode}{hyperref}
\PassOptionsToPackage{hyphens}{url}
%
\documentclass[
]{article}
\usepackage{amsmath,amssymb}
\usepackage{lmodern}
\usepackage{iftex}
\ifPDFTeX
  \usepackage[T1]{fontenc}
  \usepackage[utf8]{inputenc}
  \usepackage{textcomp} % provide euro and other symbols
\else % if luatex or xetex
  \usepackage{unicode-math}
  \defaultfontfeatures{Scale=MatchLowercase}
  \defaultfontfeatures[\rmfamily]{Ligatures=TeX,Scale=1}
\fi
% Use upquote if available, for straight quotes in verbatim environments
\IfFileExists{upquote.sty}{\usepackage{upquote}}{}
\IfFileExists{microtype.sty}{% use microtype if available
  \usepackage[]{microtype}
  \UseMicrotypeSet[protrusion]{basicmath} % disable protrusion for tt fonts
}{}
\makeatletter
\@ifundefined{KOMAClassName}{% if non-KOMA class
  \IfFileExists{parskip.sty}{%
    \usepackage{parskip}
  }{% else
    \setlength{\parindent}{0pt}
    \setlength{\parskip}{6pt plus 2pt minus 1pt}}
}{% if KOMA class
  \KOMAoptions{parskip=half}}
\makeatother
\usepackage{xcolor}
\usepackage[margin=1in]{geometry}
\usepackage{color}
\usepackage{fancyvrb}
\newcommand{\VerbBar}{|}
\newcommand{\VERB}{\Verb[commandchars=\\\{\}]}
\DefineVerbatimEnvironment{Highlighting}{Verbatim}{commandchars=\\\{\}}
% Add ',fontsize=\small' for more characters per line
\usepackage{framed}
\definecolor{shadecolor}{RGB}{248,248,248}
\newenvironment{Shaded}{\begin{snugshade}}{\end{snugshade}}
\newcommand{\AlertTok}[1]{\textcolor[rgb]{0.94,0.16,0.16}{#1}}
\newcommand{\AnnotationTok}[1]{\textcolor[rgb]{0.56,0.35,0.01}{\textbf{\textit{#1}}}}
\newcommand{\AttributeTok}[1]{\textcolor[rgb]{0.77,0.63,0.00}{#1}}
\newcommand{\BaseNTok}[1]{\textcolor[rgb]{0.00,0.00,0.81}{#1}}
\newcommand{\BuiltInTok}[1]{#1}
\newcommand{\CharTok}[1]{\textcolor[rgb]{0.31,0.60,0.02}{#1}}
\newcommand{\CommentTok}[1]{\textcolor[rgb]{0.56,0.35,0.01}{\textit{#1}}}
\newcommand{\CommentVarTok}[1]{\textcolor[rgb]{0.56,0.35,0.01}{\textbf{\textit{#1}}}}
\newcommand{\ConstantTok}[1]{\textcolor[rgb]{0.00,0.00,0.00}{#1}}
\newcommand{\ControlFlowTok}[1]{\textcolor[rgb]{0.13,0.29,0.53}{\textbf{#1}}}
\newcommand{\DataTypeTok}[1]{\textcolor[rgb]{0.13,0.29,0.53}{#1}}
\newcommand{\DecValTok}[1]{\textcolor[rgb]{0.00,0.00,0.81}{#1}}
\newcommand{\DocumentationTok}[1]{\textcolor[rgb]{0.56,0.35,0.01}{\textbf{\textit{#1}}}}
\newcommand{\ErrorTok}[1]{\textcolor[rgb]{0.64,0.00,0.00}{\textbf{#1}}}
\newcommand{\ExtensionTok}[1]{#1}
\newcommand{\FloatTok}[1]{\textcolor[rgb]{0.00,0.00,0.81}{#1}}
\newcommand{\FunctionTok}[1]{\textcolor[rgb]{0.00,0.00,0.00}{#1}}
\newcommand{\ImportTok}[1]{#1}
\newcommand{\InformationTok}[1]{\textcolor[rgb]{0.56,0.35,0.01}{\textbf{\textit{#1}}}}
\newcommand{\KeywordTok}[1]{\textcolor[rgb]{0.13,0.29,0.53}{\textbf{#1}}}
\newcommand{\NormalTok}[1]{#1}
\newcommand{\OperatorTok}[1]{\textcolor[rgb]{0.81,0.36,0.00}{\textbf{#1}}}
\newcommand{\OtherTok}[1]{\textcolor[rgb]{0.56,0.35,0.01}{#1}}
\newcommand{\PreprocessorTok}[1]{\textcolor[rgb]{0.56,0.35,0.01}{\textit{#1}}}
\newcommand{\RegionMarkerTok}[1]{#1}
\newcommand{\SpecialCharTok}[1]{\textcolor[rgb]{0.00,0.00,0.00}{#1}}
\newcommand{\SpecialStringTok}[1]{\textcolor[rgb]{0.31,0.60,0.02}{#1}}
\newcommand{\StringTok}[1]{\textcolor[rgb]{0.31,0.60,0.02}{#1}}
\newcommand{\VariableTok}[1]{\textcolor[rgb]{0.00,0.00,0.00}{#1}}
\newcommand{\VerbatimStringTok}[1]{\textcolor[rgb]{0.31,0.60,0.02}{#1}}
\newcommand{\WarningTok}[1]{\textcolor[rgb]{0.56,0.35,0.01}{\textbf{\textit{#1}}}}
\usepackage{longtable,booktabs,array}
\usepackage{calc} % for calculating minipage widths
% Correct order of tables after \paragraph or \subparagraph
\usepackage{etoolbox}
\makeatletter
\patchcmd\longtable{\par}{\if@noskipsec\mbox{}\fi\par}{}{}
\makeatother
% Allow footnotes in longtable head/foot
\IfFileExists{footnotehyper.sty}{\usepackage{footnotehyper}}{\usepackage{footnote}}
\makesavenoteenv{longtable}
\usepackage{graphicx}
\makeatletter
\def\maxwidth{\ifdim\Gin@nat@width>\linewidth\linewidth\else\Gin@nat@width\fi}
\def\maxheight{\ifdim\Gin@nat@height>\textheight\textheight\else\Gin@nat@height\fi}
\makeatother
% Scale images if necessary, so that they will not overflow the page
% margins by default, and it is still possible to overwrite the defaults
% using explicit options in \includegraphics[width, height, ...]{}
\setkeys{Gin}{width=\maxwidth,height=\maxheight,keepaspectratio}
% Set default figure placement to htbp
\makeatletter
\def\fps@figure{htbp}
\makeatother
\setlength{\emergencystretch}{3em} % prevent overfull lines
\providecommand{\tightlist}{%
  \setlength{\itemsep}{0pt}\setlength{\parskip}{0pt}}
\setcounter{secnumdepth}{-\maxdimen} % remove section numbering
\ifLuaTeX
  \usepackage{selnolig}  % disable illegal ligatures
\fi
\IfFileExists{bookmark.sty}{\usepackage{bookmark}}{\usepackage{hyperref}}
\IfFileExists{xurl.sty}{\usepackage{xurl}}{} % add URL line breaks if available
\urlstyle{same} % disable monospaced font for URLs
\hypersetup{
  pdftitle={Thema09DementiaPrediction},
  pdfauthor={Ewoud},
  hidelinks,
  pdfcreator={LaTeX via pandoc}}

\title{Thema09DementiaPrediction}
\author{Ewoud}
\date{2023-09-06}

\begin{document}
\maketitle

\hypertarget{introduction}{%
\subsection{Introduction}\label{introduction}}

Dementia is a pressing global health concern, with a significant impact
on individuals, families, and healthcare systems. Timely diagnosis and
intervention are crucial for improving the quality of life for those
affected by dementia. Advances in machine learning and healthcare
technology offer promising opportunities to enhance the accuracy and
efficiency of dementia diagnosis.

The question this research is aiming to give an answer to is: \emph{How
accurate can a machine learning model be, that predicts if a subject has
dementia using different clinical parameters?}

Our approach combines machine learning and dementia research to uncover
hidden patterns in clinical data. We will conduct an Exploratory Data
Analysis (EDA) to identify correlations with the dementia group,
assisting in feature selection and model development.

Dataset:
\url{https://www.kaggle.com/datasets/shashwatwork/dementia-prediction-dataset}

\hypertarget{codebook}{%
\subsection{Codebook}\label{codebook}}

\begin{longtable}[]{@{}
  >{\centering\arraybackslash}p{(\columnwidth - 4\tabcolsep) * \real{0.1806}}
  >{\centering\arraybackslash}p{(\columnwidth - 4\tabcolsep) * \real{0.4444}}
  >{\centering\arraybackslash}p{(\columnwidth - 4\tabcolsep) * \real{0.1667}}@{}}
\caption{Table continues below}\tabularnewline
\toprule()
\begin{minipage}[b]{\linewidth}\centering
Name
\end{minipage} & \begin{minipage}[b]{\linewidth}\centering
description
\end{minipage} & \begin{minipage}[b]{\linewidth}\centering
type
\end{minipage} \\
\midrule()
\endfirsthead
\toprule()
\begin{minipage}[b]{\linewidth}\centering
Name
\end{minipage} & \begin{minipage}[b]{\linewidth}\centering
description
\end{minipage} & \begin{minipage}[b]{\linewidth}\centering
type
\end{minipage} \\
\midrule()
\endhead
Subject.ID & Id of the patient & character \\
MRI ID & Id of MRI & character \\
Group & Converted / Demented/ Nondemented & character \\
Visit & Number of visit & character \\
MR Delay & Delay with MRI & double \\
M/F & Gender : Male / Female & character \\
Hand & Handedness & character \\
Age & Age of the subject at time of visit & double \\
EDUC & Years of education & double \\
SES & Socioeconomic status & double \\
MMSE & Mini-Mental State Examination score & double \\
CDR & Clinical Dementia Rating & double \\
eTIV & Estimated total intracranial volume & double \\
nWBV & Normalized whole-brain volume & double \\
ASF & Atlas scaling factor & double \\
\bottomrule()
\end{longtable}

\begin{longtable}[]{@{}
  >{\centering\arraybackslash}p{(\columnwidth - 0\tabcolsep) * \real{0.4583}}@{}}
\toprule()
\begin{minipage}[b]{\linewidth}\centering
value
\end{minipage} \\
\midrule()
\endhead
OAS2\_0001 - OAS2\_0186 \\
OAS2\_0001\_MRI1 - OAS2\_0186\_MRI3 \\
Converted-Demented-Nondemented \\
1-5 \\
0-2639 \\
M-F \\
R-L \\
60- 98 \\
6-23 \\
1-5 \\
4-30 \\
0.0-2.0 \\
1105-2005 \\
0.64-0.84 \\
0.87-1.59 \\
\bottomrule()
\end{longtable}

\hypertarget{description-of-some-of-the-rows}{%
\subsubsection{Description of some of the
rows}\label{description-of-some-of-the-rows}}

SES : Socioeconomic status as assessed by the Hollingshead Index of
Social Position and classified into categories from 1 (highest status)
to 5 (lowest status)

MMSE : Mini--Mental State Examination (MMSE) The Mini--Mental State
Examination (MMSE) or Folstein test is a 30-point questionnaire that is
used extensively in clinical and research settings to measure cognitive
impairment. It is commonly used in medicine and allied health to screen
for dementia. It is also used to estimate the severity and progression
of cognitive impairment and to follow the course of cognitive changes in
an individual over time; thus making it an effective way to document an
individual's response to treatment. The MMSE's purpose has been not, on
its own, to provide a diagnosis for any particular nosological entity.

Interpretations

Any score greater than or equal to 24 points (out of 30) indicates a
normal cognition. Below this, scores can indicate severe (≤9 points),
moderate (10--18 points) or mild (19--23 points) cognitive impairment.
The raw score may also need to be corrected for educational attainment
and age. That is, a maximal score of 30 points can never rule out
dementia. Low to very low scores correlate closely with the presence of
dementia, although other mental disorders can also lead to abnormal
findings on MMSE testing. The presence of purely physical problems can
also interfere with interpretation if not properly noted; for example, a
patient may be physically unable to hear or read instructions properly,
or may have a motor deficit that affects writing and drawing skills.

CDR : Clinical Dementia Rating (CDR) The CDR™ in one aspect is a 5-point
scale used to characterize six domains of cognitive and functional
performance applicable to Alzheimer disease and related dementias:
Memory, Orientation, Judgment \& Problem Solving, Community Affairs,
Home \& Hobbies, and Personal Care. The necessary information to make
each rating is obtained through a semi-structured interview of the
patient and a reliable informant or collateral source (e.g., family
member) referred to as the CDR™ Assessment Protocol.

The CDR™ Scoring Table provides descriptive anchors that guide the
clinician in making appropriate ratings based on interview data and
clinical judgment. In addition to ratings for each domain, an overall
CDR™ score may be calculated through the use of an CDR™ Scoring
Algorithm. This score is useful for characterizing and tracking a
patient's level of impairment/dementia:

0 = Normal 0.5 = Very Mild Dementia 1 = Mild Dementia 2 = Moderate
Dementia 3 = Severe Dementia

eTIV: Estimated total intracranial volume (eTIV) The ICV measure,
sometimes referred to as total intracranial volume (TIV), refers to the
estimated volume of the cranial cavity as outlined by the supratentorial
dura matter or cerebral contour when dura is not clearly detectable. ICV
is often used in studies involved with analysis of the cerebral
structure under different imaging modalities, such as Magnetic Resonance
(MR), MR and Diffusion Tensor Imaging (DTI), MR and Single-photon
Emission Computed Tomography (SPECT), Ultrasound and Computed Tomography
(CT). ICV consistency during aging makes it a reliable tool for
correction of head size variation across subjects in studies that rely
on morphological features of the brain. ICV, along with age and gender
are reported as covariates to adjust for regression analyses in
investigating progressive neurodegenerative brain disorders, such as
Alzheimer's disease, aging and cognitive impairment. ICV has also been
utilized as an independent voxel based morphometric feature to evaluate
age-related changes in the structure of premorbid brai, determine
characterizing atrophy patterns in subjects with mild cognitive
impairment (MCI) and Alzheimer's disease (AD), delineate structural
abnormalities in the white matter (WM) in schizophrenia, epilepsy, and
gauge cognitive efficacy.

nWBV : Normalized whole-brain volume, expressed as a percent of all
voxels in the atlas-masked image that are labeled as gray or white
matter by the automated tissue segmentation process

ASF: Atlas scaling factor (unitless). Computed scaling factor that
transforms native-space brain and skull to the atlas target (i.e., the
determinant of the transform matrix)

\hypertarget{cleaning}{%
\subsection{Cleaning}\label{cleaning}}

First thing to do is to clean the dataset of zero values or outliers
that can obstruct this research

\hypertarget{missing-values}{%
\subsubsection{Missing values}\label{missing-values}}

Lets filter the rows out of the dataset with 0 or na.

\begin{Shaded}
\begin{Highlighting}[]
\NormalTok{Data1\_filtered }\OtherTok{\textless{}{-}}\NormalTok{ Data1 }\SpecialCharTok{\%\textgreater{}\%} \FunctionTok{drop\_na}\NormalTok{()}
\end{Highlighting}
\end{Shaded}

19 object are filtered

\hypertarget{finding-outliers}{%
\subsubsection{finding outliers}\label{finding-outliers}}

\begin{verbatim}
##   Subject ID           MRI ID             Group              Visit              MR Delay          M/F           
##  Length:354         Length:354         Length:354         Length:354         Min.   :   0.0   Length:354        
##  Class :character   Class :character   Class :character   Class :character   1st Qu.:   0.0   Class :character  
##  Mode  :character   Mode  :character   Mode  :character   Mode  :character   Median : 559.5   Mode  :character  
##                                                                              Mean   : 601.4                     
##                                                                              3rd Qu.: 882.5                     
##                                                                              Max.   :2639.0                     
##      Hand                Age             EDUC            SES            MMSE            CDR              eTIV     
##  Length:354         Min.   :60.00   Min.   : 6.00   Min.   :1.00   Min.   : 4.00   Min.   :0.0000   Min.   :1106  
##  Class :character   1st Qu.:71.00   1st Qu.:12.00   1st Qu.:2.00   1st Qu.:27.00   1st Qu.:0.0000   1st Qu.:1358  
##  Mode  :character   Median :77.00   Median :15.00   Median :2.00   Median :29.00   Median :0.0000   Median :1470  
##                     Mean   :77.03   Mean   :14.70   Mean   :2.46   Mean   :27.41   Mean   :0.2712   Mean   :1490  
##                     3rd Qu.:82.00   3rd Qu.:16.75   3rd Qu.:3.00   3rd Qu.:30.00   3rd Qu.:0.5000   3rd Qu.:1595  
##                     Max.   :98.00   Max.   :23.00   Max.   :5.00   Max.   :30.00   Max.   :2.0000   Max.   :2004  
##       nWBV             ASF        
##  Min.   :0.6444   Min.   :0.8755  
##  1st Qu.:0.6987   1st Qu.:1.1002  
##  Median :0.7291   Median :1.1935  
##  Mean   :0.7299   Mean   :1.1938  
##  3rd Qu.:0.7569   3rd Qu.:1.2923  
##  Max.   :0.8368   Max.   :1.5873
\end{verbatim}

No visible outliers

\hypertarget{remove-colums-with-no-meaning}{%
\subsubsection{Remove colums with no
meaning}\label{remove-colums-with-no-meaning}}

\includegraphics{Thema09DementiaPrediction_files/figure-latex/unnamed-chunk-6-1.pdf}
When looking at the Handedness parameter we can see that the only
variable is R for right handed people, because everybody is right handed
we can remove the column

we can also remove the MRI id and delay because these are not parameters
that have influence on the outcome if a subject has dementia

The CDR is basically a parameter telling if a patient has dementia or
not so it will not be taking in for machine learning but testing if
algorithm is correct.

\hypertarget{testing-the-dataset}{%
\subsection{Testing the dataset}\label{testing-the-dataset}}

Second thing to do is to look at the underlying distribution and the
variation within the dataset

\hypertarget{equal-distribution}{%
\subsubsection{equal distribution}\label{equal-distribution}}

\includegraphics{Thema09DementiaPrediction_files/figure-latex/unnamed-chunk-9-1.pdf}
In figure 2 we can see that there is a nice equal distribution of the
age and gender parameters. In the dementia status group the converted
group has much less patients then the other two, in this case it doesn't
really matter because the coverted group are patients that were first
diagnosed with dementia but with a second test were labeled nondemented.
so it really is part of the non demented group but is interesting to
keep apart to see if it will be some kind of middle group

\hypertarget{finding-variation}{%
\subsection{Finding variation}\label{finding-variation}}

\includegraphics{Thema09DementiaPrediction_files/figure-latex/unnamed-chunk-10-1.pdf}
\includegraphics{Thema09DementiaPrediction_files/figure-latex/unnamed-chunk-10-2.pdf}
\includegraphics{Thema09DementiaPrediction_files/figure-latex/unnamed-chunk-10-3.pdf}
The things to look for in a boxplot is to see if the values of the
dementia groups are far apart of each other with the parameters, Because
this means that the influence of the parameters effects the groups
different and is therefor maybe a good parameters for correlation and
the machine learning model

The parameters with the most difference are: Educatie, SES, MMSE, nWBV
These parameters are mostly the ones that are going the be used to make
a model.

\hypertarget{finding-correlation}{%
\subsection{Finding correlation}\label{finding-correlation}}

The data that we are using for this project is nominal data so we need
to do a Anova test

What is a Anova test and what is the eta-sqaured?

An ANOVA (Analysis of Variance) test is used to assess statistical
differences in means among three or more groups or treatments. This type
of statistical analysis is suitable in situations where you want to
investigate whether there are significant differences between groups,
with the independent variable being a categorical variable (such as
different groups, categories, or treatments) and the dependent variable
being a continuous variable (such as measurements, scores, or values).

Eta-squared (η²) is a statistical measure used in the context of
analysis of variance (ANOVA) to quantify the proportion of variance in a
dependent variable that is attributed to the effects of an independent
variable (factor) in an experiment. It is often used to estimate the
effect size of an independent variable on the dependent variable.

Eta-squared ranges from 0 to 1 and is interpreted as follows:

η² = 0: There is no effect of the independent variable on the dependent
variable. η² ≈ 0.01: A small effect. η² ≈ 0.06: A medium effect. η² ≈
0.14: A large effect.

It's important to note that eta-squared indicates the proportion of
variance in the dependent variable explained by the independent
variable, but it does not provide information about the direction or
nature of the effect. For more detailed analyses, it may be useful to
look at other statistical measures, such as partial eta-squared, which
takes into account other factors in the model.

\#\#\#\#anova test

\begin{Shaded}
\begin{Highlighting}[]
\CommentTok{\#omzetten naar nummeric omdat het niet werkt met character }
\NormalTok{Data1\_filtered}\SpecialCharTok{$}\NormalTok{Group }\OtherTok{\textless{}{-}} \FunctionTok{as.numeric}\NormalTok{(}\FunctionTok{c}\NormalTok{(}\StringTok{"Demented"} \OtherTok{=} \StringTok{"1"}\NormalTok{, }\StringTok{"Converted"} \OtherTok{=} \StringTok{"2"}\NormalTok{, }\StringTok{"Nondemented"} \OtherTok{=} \StringTok{"3"}\NormalTok{)[Data1\_filtered}\SpecialCharTok{$}\NormalTok{Group])}
\CommentTok{\#Data1\_filtered$\textasciigrave{}M/F\textasciigrave{} \textless{}{-} as.numeric(c("M" = "1", "F" = "2")[Data1\_filtered$\textasciigrave{}M/F\textasciigrave{}])}
\CommentTok{\#Data1\_filtered$\textasciigrave{}M/F\textasciigrave{} \textless{}{-} as.numeric(c("M" = "1", "F" = "2")[Data1\_filtered$\textasciigrave{}M/F\textasciigrave{}])}
\NormalTok{Data1\_filtered}\SpecialCharTok{$}\NormalTok{Visit }\OtherTok{\textless{}{-}} \ConstantTok{NULL}

\NormalTok{one.way }\OtherTok{\textless{}{-}} \FunctionTok{aov}\NormalTok{(Group }\SpecialCharTok{\textasciitilde{}}\NormalTok{ Age }\SpecialCharTok{+}\NormalTok{ EDUC }\SpecialCharTok{+}\NormalTok{SES }\SpecialCharTok{+}\NormalTok{ MMSE }\SpecialCharTok{+}\NormalTok{ eTIV }\SpecialCharTok{+}\NormalTok{ nWBV }\SpecialCharTok{+}\NormalTok{ ASF, }\AttributeTok{data =}\NormalTok{ Data1\_filtered)}

\NormalTok{anovatest }\OtherTok{\textless{}{-}} \FunctionTok{data.frame}\NormalTok{(}\FunctionTok{unclass}\NormalTok{(}\FunctionTok{etaSquared}\NormalTok{(one.way)), }\AttributeTok{check.names =} \ConstantTok{FALSE}\NormalTok{, }\AttributeTok{stringsAsFactors =} \ConstantTok{FALSE}\NormalTok{)}
\NormalTok{anovatest }\OtherTok{\textless{}{-}} \FunctionTok{as\_tibble}\NormalTok{(anovatest)}
\NormalTok{anovanames }\OtherTok{\textless{}{-}} \FunctionTok{c}\NormalTok{(}\StringTok{"Age"}\NormalTok{,}\StringTok{"EDUC"}\NormalTok{,}\StringTok{"SES"}\NormalTok{,}\StringTok{"MMSE"}\NormalTok{,}\StringTok{"eTIV"}\NormalTok{,}\StringTok{"nWBV"}\NormalTok{,}\StringTok{"ASF"}\NormalTok{)}
\NormalTok{anovatest }\OtherTok{\textless{}{-}}\NormalTok{ anovatest }\SpecialCharTok{\%\textgreater{}\%} \FunctionTok{mutate}\NormalTok{(}\AttributeTok{varnames =}\NormalTok{ anovanames)}

\FunctionTok{ggplot}\NormalTok{(}\AttributeTok{data=}\NormalTok{anovatest, }\FunctionTok{aes}\NormalTok{(}\AttributeTok{x=}\NormalTok{varnames, }\AttributeTok{y=}\NormalTok{ eta.sq, }\AttributeTok{color =}\NormalTok{ varnames)) }\SpecialCharTok{+}
  \FunctionTok{geom\_bar}\NormalTok{(}\AttributeTok{stat=}\StringTok{"identity"}\NormalTok{, }\AttributeTok{fill =} \StringTok{\textquotesingle{}white\textquotesingle{}}\NormalTok{) }\SpecialCharTok{+}
  \FunctionTok{ggtitle}\NormalTok{(}\StringTok{"A barplot of the eta squared scores of coherents of the parameters against the dementia status, *figure 6*"}\NormalTok{) }\SpecialCharTok{+} \FunctionTok{theme}\NormalTok{(}\AttributeTok{plot.title =} \FunctionTok{element\_text}\NormalTok{(}\AttributeTok{size =} \DecValTok{7}\NormalTok{, }\AttributeTok{face =} \StringTok{"bold"}\NormalTok{))}
\end{Highlighting}
\end{Shaded}

\includegraphics{Thema09DementiaPrediction_files/figure-latex/unnamed-chunk-11-1.pdf}
The anova test above shows the eta squared score

When looking at the outcome we see that MMSE, nWBV, EDUC and Age have a
effect on the dementia status. We will be taking these parameters for
further analysis.

\#\#\#\#heatmap **Uitleg waarom heatmap en veranderen van nominal data
naar nummeric

In the article of the data set, the converted group is people that were
identified as demented but a second test confirmed that they were non
demented so it is a group that swings in the middle. If i want to
transform the nominal group of demented converted and nondemented to
numerical i can change it to 1 2 and 3 because there is a relation.

\url{https://www.sciencedirect.com/science/article/pii/S2352914819300917?via\%3Dihub}

Explaining the present MRI sessions categorization based on the current
CDR (0--2) score and total sessions of non-demented (190), demented
(146) and converted (37) were evaluated. In particular, some subjects
treated as demented at initial visit later transformed into the
non-demented managed by converted type.

With the demented groups now transformed to numerical i can make the
heatmap.

\begin{verbatim}
## # A tibble: 64 x 3
##    varnames variable      cor
##    <chr>    <chr>       <dbl>
##  1 Group    Group     1      
##  2 Group    Age       0.0428 
##  3 Group    EDUC      0.205  
##  4 Group    SES      -0.140  
##  5 Group    MMSE      0.604  
##  6 Group    eTIV      0.0165 
##  7 Group    nWBV      0.318  
##  8 Group    ASF      -0.00761
##  9 Age      Group     0.0428 
## 10 Age      Age       1      
## # i 54 more rows
\end{verbatim}

\includegraphics{Thema09DementiaPrediction_files/figure-latex/unnamed-chunk-12-1.pdf}
Wat zegt deze heatmap **

\hypertarget{principal-component-analysis}{%
\subsection{Principal component
analysis}\label{principal-component-analysis}}

** Uitleg PCA en waarom

Looking for clusters with a principal component analysis plot using the
4 most correlated parameters with dementia.
\includegraphics{Thema09DementiaPrediction_files/figure-latex/unnamed-chunk-13-1.pdf}

\end{document}
